\documentclass{article}
\usepackage{fullpage}

\begin{document}
\title{ECE750-T05: Static Analysis for Software Engineering\\Fall 2008}
\author{Patrick Lam}
\renewcommand{\today}{September 12, 2008}
\maketitle

\section*{Brief Overview}

Code review is a key technique for ensuring software quality. However,
human reviewers have limitations: limited time, limited attention
spans, and a limited understanding of the implications of a software
modification (due to interactions between parts of a potentially vast
codebase).

Computers can successfully carry out many tasks at which humans fail.
A goal of my research is to find classes of properties which are
amenable to automatic verification, using static analysis
techniques (which trace their roots to optimizing compilers). 
Many challenges exist. The most notorious
problems include the undecidability of the halting problem (which we
overcome using approximations) and the need to deal with unknown input
values. More recently, large codebases and plugin-based software
architectures pose additional challenges to static analysis.

Nevertheless, a number of static analysis techniques for software
engineering have recently been proposed and even deployed in commercial
development environments. Microsoft's Static Driver Verifier is the 
most prominent example of a static analysis technique that has escaped the
laboratory.

In this seminar course, we will first briefly explore the strengths and
limitations of static analysis techniques. These techniques
traditionally come from the compiler research community. The bulk of
this course, however, will consist of a discussion of current research papers
in the field of software verification using static analysis techniques.

\section*{General Information}

\noindent
{\bf Course Web Page:} {\tt http://patricklam.ca/sase}\\
{\bf Lectures:} Fridays, 8:30-11:20, EIT 3151\\

\noindent
{\bf Instructor:} \\

\noindent
\hspace*{2em} \begin{minipage}{.6\textwidth}
Prof. Patrick Lam\\
Office: DC2534\\
Office Hours: Thursday 2:00PM-3:00PM, or by appointment\\
Email: {\tt p.lam@ece.uwaterloo.ca}\\
\end{minipage}

\section*{Course Description}
\paragraph{Objectives}
\begin{itemize}
\item Understand the strengths and limitations of static analysis techniques.
\item Gain a familiarity with the research literature on static analysis for software verification.
\item Carry out a moderately-sized research project which implements a static
analysis and evaluates its efficacy at solving a software engineering problem.
\end{itemize}

\paragraph{Topics (a guideline).}
Overview of techniques used in static analysis, including dataflow
analysis, pointer analysis, model checking, and theorem
proving. Abstract interpretation. Type systems. Lightweight
specification languages. Applications to software engineering.

\section*{Reference Material}
Since this course is a seminar course based on published articles, your
primary source of information will be the articles that we will discuss
during the class. A list of these articles will appear on the course
webpage.

In the first lecture or two, I will give a crash course on compiler
representations and dataflow analysis.

\section*{Evaluation}
You will be expected to present two papers to the class and to
complete a course project, which includes a short presentation and may
be done in teams. Due to departmental regulations, there will also
be a final examination.

Here is the breakdown of marks:

\begin{tabular}{rl}
20\% & Paper presentations \\
10\% & Questions during paper presentations \\
20\% & Final exam \\
40\% & Project write-up \\
10\% & Project presentation \\
-5\% & Missed proposal or midterm report deadlines.
\end{tabular}

\paragraph{Presentation.} You will, collectively, present all of the 
papers that we are discussing in this class. Please let me know which
papers you have chosen by the end of the second week of classes. The
presentations will be an hour long. Be prepared to answer questions on
the papers. Also, I will expect one student to prepare a set of
questions or clarifications for each paper, to be discussed after the
paper presentation.

\paragraph{Course project.} The course project gives you an opportunity to
work on a particular application of static analysis techniques to
software engineering issues. I expect that most projects will include
an implementation component. To help you stay on track, I will expect
a project proposal by the third week of class and a mid-term project
update (1 page) by the end of October. (Start early!) Please come see me
to talk about possible projects!

Note that I will not assign a grade to the proposal or midterm
report. However, if you do not hand in a proposal on time, I will deduct 
5\% from your course mark.

\paragraph{Final examination.} The open-notes final examination will ask about
your ability to summarize the key points of some of the papers that we'll
discuss.

\paragraph{Grading.} I will assign letter grades for the various components
of the course and convert them to numbers as follows.

\begin{tabular}{cc}
A & 95\% \\
B & 80\% \\
C & 60\% \\
\end{tabular}

I will also use plusses and minusses as appropriate.

\end{document}
