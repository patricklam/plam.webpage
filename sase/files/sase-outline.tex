\documentclass{article}
\usepackage{fullpage}
\usepackage{url}

\begin{document}
\title{Static Analysis for Software Engineering\\ECE750-T5, Spring 2013}
\author{Patrick Lam}
\renewcommand{\today}{}
\maketitle
\vspace*{-2em}

Code review is a key technique for ensuring software quality. However,
human reviewers have limitations: limited time, limited attention
spans, and a limited understanding of the implications of a software
modification (due to interactions between parts of a potentially vast
codebase).

Computers can successfully carry out many tasks at which humans fail.
A goal of my research is to find classes of properties which are
amenable to automatic verification, using static analysis
techniques (which trace their roots to optimizing compilers). 
Many challenges exist. The most notorious
problems include the undecidability of the halting problem (which we
overcome using approximations) and the need to deal with unknown input
values. More recently, large codebases and plugin-based software
architectures pose additional challenges to static analysis.

Nevertheless, a number of static analysis techniques for software
engineering have recently been proposed and even deployed in commercial
development environments. Microsoft's Static Driver Verifier is the 
most prominent example of a static analysis technique that has escaped the
laboratory.

In this seminar course, we will first briefly explore the strengths and
limitations of program analysis techniques. These techniques
traditionally come from the compiler research community. The bulk of
this course, however, will consist of a discussion of current research papers
in the field of software verification using program analysis techniques.

\paragraph{Objectives.} After this course, you will:
\begin{itemize}
\item understand the strengths and limitations of program analysis
  techniques, particularly static analysis techniques;
\item be familiar with the research literature on static analysis for
  software verification.
\item have carried out a small research project implementing a program
  analysis and evaluating its efficacy at solving a software
  engineering problem.
\end{itemize}

\section*{General Information}

\noindent
{\bf Course Web Page:} {\tt http://patricklam.ca/sase}\\[0.5em]
\noindent {\bf Lectures:} Thursday, 9:30--12:30 (as of May 30; 8:30--11:30 until then.)

\vspace*{1em}\noindent
{\bf Instructor:} \\

\noindent
\hspace*{2em} \begin{minipage}{.6\textwidth}
Prof. Patrick Lam\\
Office: DC2597D\\
Office Hours: By appointment\\
Email: {\tt p.lam@ece.uwaterloo.ca}\\
Phone: Use email instead!

\end{minipage} \\[1em]


\section*{Course Description}
Overview of techniques used in static and dynamic analysis, including
dataflow analysis, type systems, and pointer analysis. Typestate
properties. Applications to software engineering (notably concurrency
and security).

Here is a week-by-week schedule:

\begin{tabular}{ll}
1--2 & Introductory material \\
3 & Applications of Type Systems \\
4 & Dynamic Heap Analysis \\
5 & Deeper Heap Analyses \\
6 & Pointer and Heap Analyses \\
7 & Test Generation via Program Analysis \\
8 & Program Transformations \\
9 & Dynamic Analysis for unsafety and overflows \\
10 & Automatic Program Repair \\
11 & Security
\end{tabular}

\section*{Reference Material}
The reference material for this course consists of my notes for the first
two lectures and the research papers that we'll discuss every week. All
papers have been posted on the authors' webpages, and I've included those links.


\section*{Evaluation}
You will be expected to present one or two papers to the class and to
complete a course project, which includes a short presentation on the
last day of class. Due to departmental regulations, there will also be
a final examination.

~\\

\begin{tabular}{lrl}
Course project & 50\% & due last day of classes \\
Project presentation & 5\% & during last class \\
Paper presentations & 20\% & throughout term \\
Final exam & 25\% & final exam period
\end{tabular}

\paragraph{Presentation.} You will present one or two of the papers (ideally two;
depends on enrollment) that we are discussing in this class. Please
let me know which papers you have chosen by May 20. The presentations
will be an hour long.  Be prepared to answer questions on the papers.

I will provide submarks for both delivery and content of your paper
and project presentations, and I'll send you timely feedback and
suggestions on how to improve your presentations. 

\paragraph{Course project.} The course project gives you an opportunity to
work on a particular application of program analysis techniques to
software engineering issues. I expect that most projects will include
an implementation component. To help you stay on track, I will expect
a project proposal by the third week of class and a project
update in week 7 (1 page). (Start early!) I will post a list of suggested
projects on the webpage. Please come see me to talk about possible
projects!

\paragraph{Final examination.} The open-notes final examination will ask you to
synthesize the information you've seen throughout the
semester. Potential questions include summarizing key points from
selected papers, or sketching out how the techniques from one paper
might apply to the problems that another paper addresses.

\paragraph{Lateness.} This is a graduate seminar, so I will offer some 
flexibility on due dates. However, this flexibility is often not in
your best interest. You must consult with me before you hand in
something late; otherwise, you will lose 5\% on the project mark for a
late project proposal or mid-term project report. The default mark for
a late project submission is 0.

\section*{Required inclusions}
{\bf Academic Integrity}: In order to maintain a culture of academic
integrity, members of the University of Waterloo community are
expected to promote honesty, trust, fairness, respect and
responsibility. [Check \url{www.uwaterloo.ca/academicintegrity/} for more
  information.]

\vspace*{1em}\noindent
{\bf Grievance}: A student who believes that a decision affecting some
aspect of his/her university life has been unfair or unreasonable may
have grounds for initiating a grievance. Read Policy 70, Student
Petitions and Grievances, Section 4,
\url{www.adm.uwaterloo.ca/infosec/Policies/policy70.htm}.  When in doubt
please be certain to contact the department’s administrative assistant
who will provide further assistance.

\vspace*{1em}\noindent
{\bf Discipline}: A student is expected to know what constitutes academic
integrity [check \url{www.uwaterloo.ca/academicintegrity/}] to avoid
committing an academic offence, and to take responsibility for his/her
actions. A student who is unsure whether an action constitutes an
offence, or who needs help in learning how to avoid offences (e.g.,
plagiarism, cheating) or about “rules” for group work/collaboration
should seek guidance from the course instructor, academic advisor, or
the undergraduate Associate Dean. For information on categories of
offences and types of penalties, students should refer to Policy 71,
Student Discipline,
\url{www.adm.uwaterloo.ca/infosec/Policies/policy71.htm}. For typical
penalties check Guidelines for the Assessment of Penalties,
www.adm.uwaterloo.ca/infosec/guidelines/penaltyguidelines.htm.

\vspace*{1em}\noindent
{\bf Appeals}: A decision made or penalty imposed under Policy 70 (Student
Petitions and Grievances) (other than a petition) or Policy 71
(Student Discipline) may be appealed if there is a ground. A student
who believes he/she has a ground for an appeal should refer to Policy
72 (Student Appeals)
\url{www.adm.uwaterloo.ca/infosec/Policies/policy72.htm}.  Note for Students
with Disabilities: The Office for Persons with Disabilities (OPD),
located in Needles Hall, Room 1132, collaborates with all academic
departments to arrange appropriate accommodations for students with
disabilities without compromising the academic integrity of the
curriculum. If you require academic accommodations to lessen the
impact of your disability, please register with the OPD at the
beginning of each academic term.

\end{document}

