\documentclass{beamer}

\usetheme{Darmstadt}

%\includeonlyframes{current}

\usepackage{times}
\usefonttheme{structurebold}

\usepackage{listings}
\usepackage{pgf}
\usepackage{tikz}
\usepackage{alltt}
\usepackage[normalem]{ulem}
\usetikzlibrary{arrows}
\usetikzlibrary{automata}
\usetikzlibrary{shapes}
\usepackage{amsmath,amssymb}
\usepackage{rotating}
\usepackage{array}

%\setbeamercovered{dynamic}

\title{Lecture 1: Compiler Fundamentals}

\author{Patrick Lam}
\date{September 12, 2008}

\colorlet{redshaded}{red!25!bg}
\colorlet{shaded}{black!25!bg}
\colorlet{shadedshaded}{black!10!bg}
\colorlet{blackshaded}{black!40!bg}

\colorlet{darkred}{red!80!black}
\colorlet{darkblue}{blue!80!black}
\colorlet{darkgreen}{green!80!black}

\newcommand{\rot}[1]{\rotatebox{90}{\mbox{#1}}}
\newcommand{\gray}[1]{\mbox{#1}}
\newtheorem{prop}{Proposition}


\begin{document}

\begin{frame}
  \titlepage
\end{frame}

\section{Example}

\lstset{%
 basicstyle=\ttfamily
}

\begin{frame}[fragile]
  \frametitle{Example Method}
  \begin{lstlisting}
    public int f(int x) {
      int y, z;

      y = x;
      z = 1;
      while (y > 1) {
        z = z * y;
        y = y - 1;
      }
      y = 0;
      return z;
    }
  \end{lstlisting}

\end{frame}

\section{Abstract Syntax Trees}

\begin{frame}
\frametitle{Abstract Syntax Tree}
\begin{center}
\begin{tikzpicture} [every node/.style={fill=red!60,draw=black,rectangle},
                     sibling distance=5.8em]
\node {method {\tt f}}
  child {node {\tt y = z}}
  child {node {\tt z = 1}}
  child {node {\tt while (y>1)}
    child {node {\tt z = z*y}}
    child {node {\tt y = y-1}}}
  child {node {\tt y = 0}}
  child {node {\tt return z}};
\end{tikzpicture}
\end{center}
\end{frame}

\section{Control Flow Graphs}

\begin{frame}
\frametitle{Control Flow Graph}
\begin{center}
\begin{tikzpicture} [auto,
                     row sep=.8em]
\matrix [every node/.style={draw=black,fill=blue!60,rectangle}]
{
  \node (yx) { \tt y = x}; \\
  \node (z1) { \tt z = 1}; \\
  \node (if) { \tt if (y>1)}; \\
  \node (zz) { \tt z = z*y}; \\
  \node (ym) { \tt y = y-1}; \\
  \node (y0) { \tt y = 0}; \\
  \node (rz) { \tt return z}; \\
};

\begin{scope}[->,thick,>=latex]
  \draw (yx) -- (z1);
  \draw (z1) -- (if);
  \draw (if) -- (zz);
  \draw (zz) -- (ym);
  \draw (ym.west) to [bend left=85] (if.west);
  \draw (if.south) .. controls (2,0.5) and (2,-1.5) .. (y0.north);
  \draw (y0) -- (rz);
\end{scope}
\end{tikzpicture}
\end{center}
\end{frame}

\begin{frame}
\frametitle{Three-Address Code}

\begin{center}
{\tt x = f(a+b*c, g(d)) }
\end{center}

\begin{center}
\small
\begin{tabular}[b]{m{8em}cm{12em}}
3-Address Code & \quad vs. \quad & \begin{center}Expression Tree\end{center} \\
\begin{minipage}{8em} \tt 
t0 = b*c \\
t1 = a+t0 \\
t2 = g(d) \\
x = f(t1, t2)
\end{minipage} && \begin{center} \begin{tikzpicture} [level distance=7mm]
\node {{\tt =}}
  child {node {\tt x}}
  child {node {\tt f}
    child {node {\tt +}
      child {node {\tt a}}
        child {node {\tt *}
          child {node {\tt b}}
          child {node {\tt c}}}}
    child {node {\tt g}
      child {node {\tt d}}}};
\end{tikzpicture}
\end{center}

\end{tabular}
\end{center}

\end{frame}

\begin{frame}
\frametitle{CFG on Basic Blocks}
\begin{center}
\begin{tikzpicture} [auto,
                     row sep=.8em]
\matrix [every node/.style={draw=black,fill=blue!60,rectangle}]
{
  \node (yx) { \tt \begin{minipage}{4em} y = x \\ z = 1 \end{minipage}}; \\
  \node (if) { \tt if (y>1)}; \\
  \node (zz) { \tt \begin{minipage}{5em} z = z*y \\ y = y-1 \end{minipage}}; \\
  \node (y0) { \tt \begin{minipage}{5em} y = 0\\ return z \end{minipage}}; \\
};

\begin{scope}[->,thick,>=latex]
  \draw (yx) -- (if);
  \draw (if) -- (zz);
  \draw (zz.west) to [bend left=85] (if.west);
  \draw (if.south) .. controls (2,0.5) and (2,-1.5) .. (y0.north);
\end{scope}
\end{tikzpicture}
\end{center}
\end{frame}

\begin{frame}
\frametitle{Basic Block Definition}

A \alert{basic block} is a sequence of statements with no jumps into
or out of the block.
\end{frame}

\section{Dataflow Analysis I}

\begin{frame}
  \frametitle{Dataflow Analysis in Brief}

\Large
Set up an abstract domain and calculate the effect of each program
statement with respect to your abstract domain, until the
fixed point.

\end{frame}

\begin{frame}
  \frametitle{Analysis 1: Reaching Definitions}
\begin{center}
Q: Which definitions reach a given variable use?

\begin{tikzpicture} [auto,
                     row sep=.8em]
\matrix [every node/.style={draw=black,fill=blue!60,rectangle}]
{
  \node (yx) { \tt 1: y = x}; \\
  \node (z1) { \tt 2: z = 1}; \\
  \node (if) { \tt 3: if (y\only<2>{$^{1}$}\only<3>{$^{1,5}$}>1)}; \\
  \node (zz) { \tt 4: z = z\only<2>{$^{2}$}\only<3>{$^{2,4}$}*y\only<2>{$^{1}$}\only<3>{$^{1,5}$}}; \\
  \node (ym) { \tt 5: y = y\only<2>{$^{1}$}\only<3>{$^{1,5}$}-1}; \\
  \node (y0) { \tt 6: y = 0}; \\
  \node (rz) { \tt 7: return z\only<3>{$^{4}$}}; \\
};

\begin{scope}[->,thick,>=latex]
  \draw (yx) -- (z1);
  \draw (z1) -- (if);
  \draw (if) -- (zz);
  \draw (zz) -- (ym);
  \draw (ym.west) to [bend left=85] (if.west);
  \draw (if.south) .. controls (3,0.75) and (3,-1.5) .. (y0.north);
  \draw (y0) -- (rz);
\end{scope}
\end{tikzpicture}
\end{center}
\end{frame}

\begin{frame}
  \frametitle{Setting up a Dataflow Analysis}

Six steps:

\begin{enumerate}
\item What is your problem?
\item Forward or backward?
\item What's in your dataflow sets?
\item Merge: union or intersection?
\item What are your transfer functions?
\item What are the initial values?
\end{enumerate}
\end{frame}

\begin{frame}
  \frametitle{Reaching Definitions Problem Statement}

A definition $d$ of variable $v$ \alert{reaches} a use $u$ if
there exists a path of control-flow edges from $d$ to $u$ that
does not contain any redefinitions of $v$.

\end{frame}

\begin{frame}
  \frametitle{Reaching Definitions: A Forward Analysis}

We move information \alert{forward} through the CFG.

\begin{center}
\begin{tikzpicture} [auto,
                     row sep=.8em]
\matrix [every node/.style={draw=black,fill=blue!60,rectangle}]
{
  \node (yx) { \tt 1: y = x}; \\
  \node (z1) { \tt 2: \alert<2>{z} = 1}; \\
  \node (if) { \tt 3: if (y>1)}; \\
  \node (zz) { \tt 4: z = \alert<2>{z}*y}; \\
  \node (ym) { \tt 5: y = y-1}; \\
  \node (y0) { \tt 6: y = 0}; \\
  \node (rz) { \tt 7: return z}; \\
};

\begin{scope}[->,thick,>=latex]
  \draw (yx) -- (z1);
  \draw (z1) -- (if);
  \draw (if) -- (zz);
  \draw (zz) -- (ym);
  \draw (ym.west) to [bend left=85] (if.west);
  \draw (if.south) .. controls (3,0.75) and (3,-1.5) .. (y0.north);
  \draw (y0) -- (rz);
\end{scope}

\draw [very thick, red] (-3, 2.8) -- (-3, -2);
\draw [very thick, red] (-3.1, 2.8) -- (-3.1, -2);
\draw [very thick, red] (-2.85, -1.8) -- (-3.05, -2.1);
\draw [very thick, red] (-3.3, -1.8) -- (-3.05, -2.1);
\end{tikzpicture}
\end{center}

\end{frame}

\begin{frame}
  \frametitle{Reaching Definitions: Abstraction}

Keep a list of definitions for each variable.

\begin{center}
\begin{tikzpicture} [auto,
                     row sep=.8em]
\matrix [every node/.style={draw=black,fill=blue!60,rectangle}]
{
  \node (yx) { \tt 1: y = x}; \\
  \node (z1) { \tt 2: z = 1}; \\
  \node (if) { \tt 3: if (y>1)}; \\
  \node (zz) { \tt 4: z = z*y}; \\
  \node (ym) { \tt 5: y = y-1}; \\
  \node (y0) { \tt 6: y = 0}; \\
  \node (rz) { \tt 7: return z}; \\
};
\node [red,right] at (yx.south east) {\small y: (1)};
\node [red,right] at (z1.south east) {\small y: (1); z: (2)};
\node [red,right] at (if.south east) {\small y: (1),(5); z: (2),(4)};
\node [red,right] at (zz.south east) {\small \qquad ~ y: (1),(5); z: (4)};
\node [red,right] at (ym.south east) {\small \qquad  y: (5); z: (4)};
\node [red,right] at (y0.south east) {\small y: (6); z: (2),(4)};

\begin{scope}[->,thick,>=latex]
  \draw (yx) -- (z1);
  \draw (z1) -- (if);
  \draw (if) -- (zz);
  \draw (zz) -- (ym);
  \draw (ym.west) to [bend left=85] (if.west);
  \draw (if.south) .. controls (3,0.75) and (3,-1.5) .. (y0.north);
  \draw (y0) -- (rz);
\end{scope}
\end{tikzpicture}
\end{center}

\end{frame}

\begin{frame}
  \frametitle{Reaching Definitions: Merge Operator}

A definition reaches if \alert{any} path exists from def to use: \alert{union}.

\begin{center}
\begin{tikzpicture} [auto,
                     row sep=.8em]
\matrix [every node/.style={draw=black,fill=blue!60,rectangle}]
{
  \node (yx) { \tt 1: y = x}; \\
  \node (z1) { \tt 2: z = 1}; \\
  \node (if) { \tt 3: if (y>1)}; \\
  \node (zz) { \tt 4: z = z*y}; \\
  \node (ym) { \tt 5: y = y-1}; \\
  \node (y0) { \tt 6: y = 0}; \\
  \node (rz) { \tt 7: return z}; \\
};
\node [red,right] at (if.north east) {\small y: (1),(5); z: (2),(4)};

\begin{scope}[->,thick,>=latex]
  \draw (yx) -- (z1);
  \draw (z1) -- (if);
  \draw (if) -- (zz);
  \draw (zz) -- (ym);
  \draw (ym.west) to [bend left=85] (if.west);
  \draw (if.south) .. controls (3,0.75) and (3,-1.5) .. (y0.north);
  \draw (y0) -- (rz);
\end{scope}
\end{tikzpicture}
\end{center}

\end{frame}

\begin{frame}
  \frametitle{Reaching Definitions: Transfer Functions}

For a some-paths analysis like reaching definitions:

\[ \mbox{out}(s) = \left( \bigcup_{i \in \mbox{\scriptsize preds}(s)} 
\mbox{out}(i) - \mbox{kill}(s) \right) \cup \mbox{gen}(s)
\]
\end{frame}

\begin{frame}
  \frametitle{Reaching Definitions: Kill Sets}

At an assignment statement,
\[ s: v = \mbox{RHS}, \]
we kill all extant definitions for variable $v$:
\[ v:*. \]
\end{frame}

\begin{frame}
  \frametitle{Reaching Definitions: Gen Sets}

At an assignment statement,
\[ s: v = \mbox{RHS}, \]
we generate a new definition $s$ for variable $v$:
\[ v:s. \]
\end{frame}

\begin{frame}
  \frametitle{Reaching Definitions: Initial Values}

\begin{itemize}
\item At the beginning of the procedure, no definitions reach any
variables; for all variables $v$, we use $\emptyset$.
\item At all other program points $p$, we start by assuming that no
definitions reach $p$ either: also use $\emptyset$.
\end{itemize}
\end{frame}

\begin{frame}
  \frametitle{Dataflow Analysis Discussion}

Six steps:

\begin{enumerate}
\item What is your problem?
\item Forward or backward?
\item What's in your dataflow sets?
\item Merge: union or intersection?
\item What are your transfer functions?
\item What are the initial values?
\end{enumerate}
\end{frame}

\begin{frame}[fragile]
  \frametitle{Forward Analysis: Reaching Defs}
  \begin{alltt}
    public int f(int x) \{
      int y, z;

      y = x;
\( \downarrow \)   \alert{z} = 1; 
      while (y > 1) \{
        z = \alert{z} * y;
        y = y - 1;
      \}
      y = 0;
      return z;
    \}
  \end{alltt}

\end{frame}

\begin{frame}[fragile]
  \frametitle{Backward Analysis: Live Locals}
  \begin{alltt}
    public int f(int x) \{
      int y, z;

      y = x;                   // \{ y \}
      z = 1;                   // \{ y, z \}
      while (y > 1) \{
        z = z * y;             // \{ z \}
        y = y - 1;             // \{ y, z \}
      \}
      y = 0;                   // \{ z \}
      return z;
\( \uparrow \) \}
  \end{alltt}

\end{frame}

\begin{frame}
  \frametitle{Dataflow Sets}

Some more examples:
\begin{itemize}
\item ``$x$ points to {\tt null} / non-{\tt null} / don't-know''
\item ``$y$ is positive / negative / zero / don't-know''
\end{itemize}

\vspace*{1em}
$\perp$ (``bottom'') represents ``don't know''---no information yet.

\vspace*{1em}
$\top$ (``top'') represents ``overdetermined''---e.g. our analysis wants one
precise answer (e.g. constant propagation: $y=5$) and we have more than one answer ($y=5$ \emph{and} $y=3$).
\end{frame}

\begin{frame}
  \frametitle{Merge operator}

\Large
$\bigcup$: some-path analysis\\ $\qquad \qquad$ (e.g. reaching defs)

\vspace*{1em}

$\bigcap$: all-paths analysis\\ $\qquad \qquad$ (e.g. available expressions)

\end{frame}

\begin{frame}
  \frametitle{Transfer functions}

\Large

  Assignment statements and invoke statements are most interesting.

\end{frame}

\begin{frame}
  \frametitle{Initial Values}

\Large

At entry point, need conservative underapproximation.

\vspace*{1em}

At other points, use overapproximation; it gets refined later.

\end{frame}

\begin{frame}

  \frametitle{Beware}

\Huge
Dataflow analysis is subtle!

\end{frame}

\section{Alternatives}

\begin{frame}
  \frametitle{Alternatives}

  \huge 
  \begin{itemize}
  \item Constraint Systems
  \item Type and Effect Systems
  \end{itemize}
\end{frame}

\begin{frame}
  \frametitle{Constraint Systems}

\begin{tabular}{cc}
\begin{minipage}{.4\textwidth}
\begin{center}
\begin{tikzpicture} [auto,
                     row sep=.8em]
\matrix [every node/.style={draw=black,fill=blue!60,rectangle}]
{
  \node (yx) { \tt y = x}; \\
  \node (z1) { \tt z = 1}; \\
  \node (if) { \tt if (y>1)}; \\
  \node (zz) { \tt z = z*y}; \\
  \node (ym) { \tt y = y-1}; \\
  \node (y0) { \tt y = 0}; \\
  \node (rz) { \tt return z}; \\
};

\begin{scope}[->,thick,>=latex]
  \draw (yx) -- (z1);
  \draw (z1) -- (if);
  \draw (if) -- (zz);
  \draw (zz) -- (ym);
  \draw (ym.west) to [bend left=85] (if.west);
  \draw (if.south) .. controls (2,0.5) and (2,-1.5) .. (y0.north);
  \draw (y0) -- (rz);
\end{scope}
\end{tikzpicture}
\end{center}
\end{minipage}
&
\begin{minipage}{.4\textwidth}
\begin{eqnarray*}
\mbox{RD}_{\mbox{\scriptsize exit}}(1) &\supseteq& \{ (y, 1) \} \\
\mbox{RD}_{\mbox{\scriptsize entry}}(3) &\supseteq& \mbox{RD}_{\mbox{\scriptsize exit}}(2) \\
\mbox{RD}_{\mbox{\scriptsize entry}}(3) &\supseteq& \mbox{RD}_{\mbox{\scriptsize exit}}(5) \\
\end{eqnarray*}
\end{minipage}
\end{tabular}


\end{frame}

\section{Dataflow Analysis II}

\begin{frame}
  \frametitle{Example 2: Available Expressions}

Which expressions have been computed and not changed since?

\begin{center}
\begin{tikzpicture} [auto,
                     row sep=.8em]
\matrix [every node/.style={draw=black,fill=blue!60,rectangle}]
{
  \node (yx) { \tt 1: y = x}; \\
  \node (z1) { \tt 2: z = 1}; \\
  \node (if) { \tt 3: if (y>1)}; \\
  \node (zz) { \tt 4: z = z*y}; \\
  \node (ym) { \tt 5: y = y-1}; \\
  \node (y0) { \tt 6: y = 0}; \\
  \node (rz) { \tt 7: return z}; \\
};
\node [red,right] at (zz.south east) {\small \qquad \{z*y\}};
\node [red,right] at (ym.south east) {\small \qquad \{y-1\}};

\begin{scope}[->,thick,>=latex]
  \draw (yx) -- (z1);
  \draw (z1) -- (if);
  \draw (if) -- (zz);
  \draw (zz) -- (ym);
  \draw (ym.west) to [bend left=85] (if.west);
  \draw (if.south) .. controls (2.5,0.75) and (2.5,-1.5) .. (y0.north);
  \draw (y0) -- (rz);
\end{scope}
\end{tikzpicture}
\end{center}

\end{frame}

\begin{frame}
  \frametitle{Available Expressions Problem Statement}

An expression {\tt v1 op v2} is \alert{available} at statement $t$ if it
has been computed at statement $s$ and all paths from $s$ to $t$ have
no redefinitions of {\tt v1} or {\tt v2}.

\end{frame}

\begin{frame}
  \frametitle{Available Expressions is a Forward Analysis}

Expressions are available if they've already been computed.

\begin{center}
\begin{tikzpicture} [auto,
                     row sep=.8em]
\matrix [every node/.style={draw=black,fill=blue!60,rectangle}]
{
  \node (yx) { \tt 1: y = x}; \\
  \node (z1) { \tt 2: z = 1}; \\
  \node (if) { \tt 3: if (y>1)}; \\
  \node (zz) { \tt 4: z = z*y}; \\
  \node (ym) { \tt 5: y = y-1}; \\
  \node (y0) { \tt 6: y = 0}; \\
  \node (rz) { \tt 7: return z}; \\
};

\begin{scope}[->,thick,>=latex]
  \draw (yx) -- (z1);
  \draw (z1) -- (if);
  \draw (if) -- (zz);
  \draw (zz) -- (ym);
  \draw (ym.west) to [bend left=85] (if.west);
  \draw (if.south) .. controls (3,0.75) and (3,-1.5) .. (y0.north);
  \draw (y0) -- (rz);
\end{scope}

\draw [very thick, red] (-3, 2.8) -- (-3, -2);
\draw [very thick, red] (-3.1, 2.8) -- (-3.1, -2);
\draw [very thick, red] (-2.85, -1.8) -- (-3.05, -2.1);
\draw [very thick, red] (-3.3, -1.8) -- (-3.05, -2.1);
\end{tikzpicture}
\end{center}

\end{frame}

\begin{frame}
  \frametitle{Available Expressions: Abstraction}

Sets of expressions.

\begin{center}
\begin{tikzpicture} [auto,
                     row sep=.8em]
\matrix [every node/.style={draw=black,fill=blue!60,rectangle}]
{
  \node (yx) { \tt 1: y = x}; \\
  \node (z1) { \tt 2: z = 1}; \\
  \node (if) { \tt 3: if (y>1)}; \\
  \node (zz) { \tt 4: z = z*y}; \\
  \node (ym) { \tt 5: y = y-1}; \\
  \node (y0) { \tt 6: y = 0}; \\
  \node (rz) { \tt 7: return z}; \\
};
\node [red,right] at (zz.south east) {\small \qquad \{z*y\}};
\node [red,right] at (ym.south east) {\small \qquad \{y-1\}};

\begin{scope}[->,thick,>=latex]
  \draw (yx) -- (z1);
  \draw (z1) -- (if);
  \draw (if) -- (zz);
  \draw (zz) -- (ym);
  \draw (ym.west) to [bend left=85] (if.west);
  \draw (if.south) .. controls (2.5,0.75) and (2.5,-1.5) .. (y0.north);
  \draw (y0) -- (rz);
\end{scope}
\end{tikzpicture}
\end{center}

\end{frame}

\begin{frame}
  \frametitle{Available Expressions: Merge Operator}
An expression must be available on \alert{all} paths: \alert{intersection}.

\begin{center}
\begin{tikzpicture} [auto,
                     row sep=.8em]
\matrix [every node/.style={draw=black,fill=blue!60,rectangle}]
{
  \node (yx) { \tt 1: y = x}; \\
  \node (z1) { \tt 2: z = 1}; \\
  \node (if) { \tt 3: if (y>1)}; \\
  \node (zz) { \tt 4: z = z*y}; \\
  \node (ym) { \tt 5: y = y-1}; \\
  \node (y0) { \tt 6: y = 0}; \\
  \node (rz) { \tt 7: return z}; \\
};

\begin{scope}[->,thick,>=latex]
  \draw (yx) -- (z1);
  \draw (z1) -- (if);
  \draw (if) -- (zz);
  \draw (zz) -- (ym);
  \draw (ym.west) to [bend left=85] (if.west);
  \draw (if.south) .. controls (3,0.75) and (3,-1.5) .. (y0.north);
  \draw (y0) -- (rz);
\end{scope}
\end{tikzpicture}
\end{center}

\end{frame}

\begin{frame}
  \frametitle{Available Expressions: Transfer Functions}

For an all-paths analysis like available expressions:

\[ \mbox{out}(s) = \left( \bigcap_{i \in \mbox{\scriptsize preds}(s)} 
\mbox{out}(i) - \mbox{kill}(s) \right) \cup \mbox{gen}(s)
\]
\end{frame}

\begin{frame}
  \frametitle{Available Expressions: Kill Sets}

At an assignment statement,
\[ s: v = \mbox{RHS}, \]
we kill all expressions containing $v$,
\[ \mbox{e.g.~~} v + q, v * z, v.f \]
\end{frame}

\begin{frame}
  \frametitle{Available Expressions: Gen Sets}

At a statement,
\[ s: \cdots = \mbox{\tt v1 op v2}, \]
we generate the expression {\tt v1 op t2}.
\end{frame}

\begin{frame}
  \frametitle{Available Expressions: Initial Values}

\Large
\begin{itemize}
\item At entry points, no expressions are available; use $\emptyset$.
\item At all other program points, assume all expressions available; use $\top$.
\end{itemize}
\end{frame}


\end{document}
